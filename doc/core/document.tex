\documentclass{article}
\usepackage{amsmath, amssymb}
\usepackage{../tutorial}



\begin{document}
Aside from the modifications to J's primitive verbs, most of the
functionality of JtoLaTeX is kept in scripts located in the \lstinline`extra`
folder. Through some trickery with \lstinline`Public_j_`, these may be accessed
just like a normal addon (see, for instance, the \lstinline`require 'array'`
statement at the top of this document's source). However, functions in the
\lstinline`core` addon are loaded automatically.

\section{Low-level verbs}
These verbs should be avoided whenever possible. They deal with the string
representation of a noun, not its actual structure. Using higher-level
verbs instead means your code will adapt sensibly to different arguments;
for instance parentheses will be resized with \lstinline`\left` and \lstinline`\right`
when surrounding a tall expression.

\lstinline`toString` converts a LaTeX noun to a string, and its inverse, \lstinline`toL`,
converts it back. \lstinline`toStrings` converts each atom of its argument to a
boxed string, and has no inverse.
\begin{lstlisting}
  (\\rightarrow |.&.toString)  =`+/ *: c,a,b
\end{lstlisting}
\[c^2=a^2+b^2 \rightarrow 2^b+2^a=2^c\]

\lstinline`concat` combines the left and right arguments, with no intervening
space.
\begin{lstlisting}
  \_var concat phi
\end{lstlisting}
\[\varphi\]

\lstinline`infix` is an adverb that concatenates \lstinline`x` and \lstinline`y` with \lstinline`u` in
the middle.
\begin{lstlisting}
  \_Long 'right'infix arrow
\end{lstlisting}
\[\Longrightarrow\]

\section{Defining things}
A few verbs are included to make the assignments at the beginning of your
document a bit easier (and more LaTeX-like).

\lstinline`is` takes a name on the left and a value on the right, assigns the
given value to the name, and returns \lstinline`i.0 0`.

\lstinline`declare` is quite a helpful adverb, and a strong candidate for the
most complicated line of code I've ever written. Given a string \lstinline`u`
which contains \lstinline`y`, \lstinline`u declare` will produce a verb---call it
\lstinline`D`---which runs its right argument through this string as \lstinline`y` and
assigns the result to the left argument. Each argument to \lstinline`D` must be
either an array of boxes or a string, which will be turned into an array
with (\lstinline`;:`). If the left argument is omitted, the right is used for both
name and value.
\begin{lstlisting}
  DeclareFunc 'textbf'  NB. I'll get to this later...
  DeclareBold =: 'textbf y' declare
  DeclareBold 'one two'
  'One Two' DeclareBold 'eleven twelve'
  list one,two,One,Two
\end{lstlisting}
\[\textbf{one},\textbf{two},\textbf{eleven},\textbf{twelve}\]

Three \lstinline`declare`-style functions are provided in \lstinline`core`. Each works
like a particular \lstinline`\` construct from JtoLaTeX's syntax.
\begin{itemize}
  \item \lstinline`DeclareConst`: like \lstinline`\_const`
  \item \lstinline`DeclareFunc`: like \lstinline`\func`
  \item \lstinline`DeclareOp`: like \lstinline`\\op`
\end{itemize}
For instance, the \lstinline`DeclareFunc 'textbf'` call above makes \lstinline`'textbf'`
into a function that applies with \lstinline`{}`.
Additionally, \lstinline`DeclareInfix` declares as an infix operator using
\lstinline`infix`.

\section{Environments}
The \lstinline`inenv` verb is a straightforward way to use LaTeX environments:
\begin{lstlisting}
  'verbatim' inenv ":i.3
\end{lstlisting}
\begin{verbatim}
  0 1 2
\end{verbatim}

To include arguments, append a box to the left argument of \lstinline`inenv` which
gives the string of arguments. You must include the brackets.
\begin{lstlisting}
  fig =: (\caption 'a variable.') concat~ mathdisp x
  ('figure';'[ht]') inenv fig
\end{lstlisting}
\begin{figure}[ht]
  \[x\]\caption{a variable.}
\end{figure}


The verbs \lstinline`mathinline` and \lstinline`mathdisp` place their arguments in inline
and display math mode (with \lstinline`$$` and \lstinline`\[\]`), respectively.

\end{document}
