\documentclass{article}
\usepackage{amsmath, amssymb}
\usepackage{array}
\usepackage{../tutorial}



\begin{document}
The \lstinline`array` addon offers a number of ways to present J arrays in LaTeX.

\section{Lists}
The verb \lstinline`list` turns its right argument into a comma-separated list.
The optional left argument gives a pair (or some other number; the first
and last are used) of characters to surround the list with.
It has the same form as the left argument to \lstinline`texs`.
Like \lstinline`texs`, \lstinline`list` will add \lstinline`\left` and \lstinline`\right` delimiters if
the list is tall.
\begin{lstlisting}
  list \([: {."1 ({:,+/)^:(<6)) 0 1
\end{lstlisting}
\[0,1,1,2,3,5\]
\begin{lstlisting}
  ('\{';'\}') list e_ i.3
\end{lstlisting}
\[\left\{e_0,e_1,e_2\right\}\]
\begin{lstlisting}
  '[()' list %3 2  NB. trick to balance parentheses
\end{lstlisting}
\[\left[\frac{1}{3},\frac{1}{2}\right)\]
For convenience, the function \lstinline`rowvec` is defined as \lstinline`'()'&list`.
\begin{lstlisting}
  rowvec 3#1
\end{lstlisting}
\[(1,1,1)\]

\section{Tables and matrices}
The verb \lstinline`totable` takes a two-dimensional array, and delimits it with
\lstinline`&` and \lstinline`\\` so it can be used in LaTeX's \lstinline`tabular` environment.
A number of functions are defined using \lstinline`totable` to give matrices with
various delimiters.
\begin{lstlisting}
  bmatrix \_omega ^ \*/~ i.3
\end{lstlisting}
\[\begin{bmatrix}
  \omega^0 & \omega^0 & \omega^0 \\
  \omega^0 & \omega^1 & \omega^2 \\
  \omega^0 & \omega^2 & \omega^4
\end{bmatrix}
\]
The following are the automatically-defined matrix commands; define a new
one with \lstinline`DeclareMatrix`. Take a look at this table's source for an
example of \lstinline`tabular` in the wild.

% arraystretch is needed to increase space between matrices.
% But the matrices are also arrays, so we have to reset it within cells.
\renewcommand{\arraystretch}{1.5}
\begin{center}
\begin{tabular}{l|>{\renewcommand{\arraystretch}{1}}c}
  Verb & Output \\
  \hline
  \lstinline`matrix` & $\begin{matrix}
    \cdot
  \end{matrix}
  $ \\
  \lstinline`bmatrix` & $\begin{bmatrix}
    \cdot
  \end{bmatrix}
  $ \\
  \lstinline`Bmatrix` & $\begin{Bmatrix}
    \cdot
  \end{Bmatrix}
  $ \\
  \lstinline`pmatrix` & $\begin{pmatrix}
    \cdot
  \end{pmatrix}
  $ \\
  \lstinline`vmatrix` & $\begin{vmatrix}
    \cdot
  \end{vmatrix}
  $ \\
  \lstinline`Vmatrix` & $\begin{Vmatrix}
    \cdot
  \end{Vmatrix}
  $
\end{tabular}

\end{center}
\renewcommand{\arraystretch}{1}

To create a LaTeX table, use the \lstinline`tabular` verb. Much like the LaTeX
environment \lstinline`tabular` requires an argument specifying the column
alignment, the J verb requires a left argument. J also allows (but does
not require) a second such argument which specifies where to place lines
vertically. This makes each \lstinline`'|'` into an \lstinline`\hline`. This argument will
automatically be extended if it is too short.
\begin{lstlisting}
  label =: ([,' ',])&.> & ('cooperates';'defects') @>
  'TAB'is ('';label A),.(label B), (list@,"0 |:) 1 3,:0 2
  ('l|cc';'.|..') tabular TAB
\end{lstlisting}
\begin{tabular}{l|cc}
   & B cooperates & B defects \\
  \hline
  A cooperates & 1,1 & 3,0 \\
  A defects & 0,3 & 2,2
\end{tabular}


\section{English lists}
The \lstinline`textlist` verb turns a list of boxed strings into an english phrase
listing the items (with a serial comma, if applicable). This is useful if,
for instance, you wish to list in a paper's text the objects given in a
table, and keep that list consistent even if you change the order of those
objects.
Similarly, \lstinline`mathlist` lists a set of math-mode objects. Each is placed
in inline math-mode separately.
\begin{lstlisting}
  textlist ;:'aluminium plexiglass copper lead'
\end{lstlisting}
aluminium, plexiglass, copper, and lead
\begin{lstlisting}
  mathlist G_2, F_4, E_]6 7 8
\end{lstlisting}
$G_2$, $F_4$, $E_6$, $E_7$, and $E_8$
\end{document}
